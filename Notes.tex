\documentclass[12pt, parskip=half]{scrreprt}
\usepackage[sexy]{evan}

\title{Notes}

\begin{document}

\maketitle

\chapter{Measure Theory}

\section{Measure Spaces}

\begin{definition}
    An \emph{algebra} $\Sigma_0$ on a set $S$ is a collection of subsets of $S$ such that
    \begin{itemize}
        \item $S\in\Sigma_0$.
        \item if $F\in\Sigma_0$, then $F^C:=S\setminus F\in\Sigma_0$.
        \item $\Sigma_0$ is closed under finite unions.
    \end{itemize}
\end{definition}

Notice that this implies that $\Sigma_0$ must also be closed under finite intersections.

\begin{definition}
    A \emph{$\sigma$-algebra} $\Sigma$ is an algebra closed under countably many unions (and thus intersections). Then, the pair $(S, \Sigma)$ is a \emph{measurable space}.
\end{definition}

\end{document}