\documentclass[12pt, parskip=half]{scrreprt}
\usepackage[sexy]{evan}

\title{Notes}

\begin{document}

\maketitle

\chapter{Measure Theory}

\section{Measure Spaces}

\begin{definition}
    An \emph{algebra} $\Sigma_0$ on a set $S$ is a collection of subsets of $S$ such that
    \begin{itemize}
        \item $S\in\Sigma_0$.
        \item if $F\in\Sigma_0$, then $F^C:=S\setminus F\in\Sigma_0$.
        \item $\Sigma_0$ is closed under finite unions.
    \end{itemize}
\end{definition}

Notice that this implies that $\Sigma_0$ must also be closed under finite intersections.

\begin{definition}
    A \emph{$\sigma$-algebra} $\Sigma$ is an algebra closed under countably many unions (and thus intersections). Then, the pair $(S, \Sigma)$ is a \emph{measurable space}.
\end{definition}

Likewise with how bases generate topologies, we may \emph{generate} an algebra from a class of subsets:

\begin{definition}
    Let $\mathcal C$ be a class of subsets of $S$. Then, let the $\sigma$-algebra \emph{generated} by $\mathcal C$ be the intersection of all $\sigma$-algebras on $S$ which superset $\mathcal C$.
\end{definition}

\begin{example}
    The $\mathcal B(S)$ \emph{Borel $\sigma$-algebra} on topological space $S$, is the $\sigma$-algebra generated by the open sets of $S$.
\end{example}

To turn a measurable space into a measure space, we must assign a measure.

\begin{definition}
    Let $\mu_0:\Sigma_0\to\mathbb R_{\ge 0}$. Then, $\mu_0$ is \emph{additive} if for any disjoint sets $F, G\in\Sigma_0$, we have that
    \[\mu_0(F\cup G)=\mu_0(F)+\mu_0(G).\]

    Likewise, denote $\mu_0$ to be \emph{countably additive} if the additive property applies for countably many added sets.
\end{definition}

\begin{definition}
    A measurable space endowed with a countably additive set function is called a \emph{measure space}.
\end{definition}

\end{document}