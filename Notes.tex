\documentclass[12pt, parskip=half]{scrreprt}
\usepackage[sexy]{evan}

\title{Notes}

\begin{document}

\maketitle

\chapter{Measure Theory}

\section{Measure Spaces}

\begin{definition}
    An \emph{algebra} $\Sigma_0$ on a set $S$ is a collection of subsets of $S$ such that
    \begin{itemize}
        \item $S\in\Sigma_0$.
        \item if $F\in\Sigma_0$, then $F^C:=S\setminus F\in\Sigma_0$.
        \item $\Sigma_0$ is closed under finite unions.
    \end{itemize}
\end{definition}

Notice that this implies that $\Sigma_0$ must also be closed under finite intersections.

\begin{definition}
    A \emph{$\sigma$-algebra} $\Sigma$ is an algebra closed under countably many unions (and thus intersections). Then, the pair $(S, \Sigma)$ is a \emph{measurable space}.
\end{definition}

Likewise with how bases generate topologies, we may \emph{generate} an algebra from a class of subsets:

\begin{definition}
    Let $\mathcal C$ be a class of subsets of $S$. Then, let the $\sigma$-algebra \emph{generated} by $\mathcal C$ be the intersection of all $\sigma$-algebras on $S$ which superset $\mathcal C$.
\end{definition}

\begin{example}
    The \emph{Borel $\sigma$-algebra} $\mathcal B(S)$ on topological space $S$, is the $\sigma$-algebra generated by the open sets of $S$.

    We also denote $\mathcal B(\mathbb R)=\mathcal B$.
\end{example}

To turn a measurable space into a measure space, we must assign a measure.

\begin{definition}
    Let $\mu_0:\Sigma_0\to\mathbb R_{\ge 0}$. Then, $\mu_0$ is \emph{additive} if for any disjoint sets $F, G\in\Sigma_0$, we have that
    \[\mu_0(F\cup G)=\mu_0(F)+\mu_0(G).\]

    Likewise, denote $\mu_0$ to be \emph{countably additive} if the additive property applies for countably many added sets.
\end{definition}

\begin{definition}
    A measurable space endowed with a countably additive set function $\mu$, denoted a \emph{measure}, is called a \emph{measure space}. If $\mu(S)=1$, then the measure is a \emph{probability measure}.
\end{definition}

We have several elementary inequalities which we may be familiar with, namely:
\begin{itemize}
    \item $\mu(A\cup B)\le \mu(A)+\mu(B)$.
    \item $\mu(\bigcup S_i)\le\sum\mu(S_i)$ for finitely many $S_i$.
\end{itemize}
Furthermore, if $\mu(S)$ is finite, then:
\begin{itemize}
    \item $\mu(A\cup B)=\mu(A)+\mu(B)-\mu(A\cap B)$.
    \item principle of inclusion-exclusion for finitely many $S_i$.
\end{itemize}

Proof of the principle of inclusion-exclusion is kind of boring and just induction to be honest.

As a result, we have the following somewhat obvious fact:
\begin{theorem}
    Let $F_n\in\Sigma$ for $n\in\mathbb N$, and let $F_n\subseteq F_{n+1}$ for all $n$ and $\bigcup F_n=F$. 

    Then, $\mu(F_n)$ approaches $\mu(F)$.
\end{theorem}

\begin{proof}
    Let $G_n$ be defined so that $G_1:=F_1$ and $G_n:=G_n\setminus G_{n-1}$ for $n\ge 2$. Then,
    \[\mu(F_n)=\mu\left(\bigcup_{i\le n}G_i\right)=\sum_{i\le n}\mu(G_i),\]
    hence
    \[\lim_{n\to\infty}\mu(F_n)=\sum\mu(G_n)=\mu(F).\]
\end{proof}

Similarly, if $F_n\supseteq F_{n+1}$ for all $n$ and $\bigcap F_n=F$, then $\mu(F_n)$ also approaches $\mu(F)$.


\begin{lemma}[Uniqueness of Extension]
    Suppose $S$ is a set, and let $\mathcal I$ be a \emph{$\pi$-system}, or a collection of subsets of $S$ closed under finite intersection. 

    Let $\Sigma:=\sigma(\mathcal I)$, and suppose $\mu_1$ and $\mu_2$ are measures on $(S,\Sigma)$ such that $\mu_1(S)=\mu_2(S)<\infty$ and for all $X\in\mathcal I$ that
    \[\mu_1(X)=\mu_2(X).\]
    Then, for all $X\in\Sigma$, we have that
    \[\mu_1(X)=\mu_2(X).\] 
\end{lemma}

\begin{theorem}[Carathedory's Extension Theorem]
    Let $S$ be a set and $\Sigma_0$ an algebra on $S$. Define $\Sigma:=\sigma(\Sigma_0)$. Then, if $\mu_0:\Sigma_0\to[0,\infty)$ is a countably additive map, then there exists a corresponding measure $\mu$ on $\Sigma$ such that for all $X\in\Sigma_0$ that
    \[\mu(X)=\mu_0(X).\]

    If $\mu_0(S)$ is finite, then this extension is unique.
\end{theorem}

\begin{example}[Lebesgue Measure]
    Let $S=(0,1]$, and for $F\subseteq S$ that $F\in\Sigma_0$ if
    \[F=(a_1,b_1]\cup \dots (a_n,b_n]\]
    for finite $n$ and $0\le a_1\le b_1\le \dots\le b_n\le 1$.

    Then, $\Sigma_0$ is an algebra on $S$ and can be uniquely extended to the measure space $\mathcal B(0,1]$ with measure $\mu$.

    If we additionally denote $\{0\}$ to have length $0$, then we have the \emph{Lebesgue measure} on $\mathcal B[0,1]$, which we commonly denote as length.
\end{example}

\section{Probability Theory}

Consider a probability space $(\Omega, \mathcal F, \mathbb P)$, where $\Omega$ is the \emph{sample space} of \emph{sample points} $\omega\in\Omega$, and $\mathcal F$ is the collection of \emph{events}. Then, we call the \emph{probability} of an event $F\in\mathcal F$ occurring to be $\mathbb P(F)$.

For example, consider the probability space of infinitely many coin flips. We'd like to show that the number of heads over the number of flips almost surely approaches $\frac 12$.

Let our sample space be
\[\Omega:=\{H,T\}^\mathbb N,\]
so that for each $\omega\in\Omega$, we have that $\omega$ is some infinite sequence with members in $\{H,T\}$ as such:
\[\omega=\{\omega_1,\dots\},\quad\omega_n\in\{H,T\}.\]

Then, we will strategically construct $\mathcal F$. Consider the set of all sample points with the $n$th flip fixed to be heads:
\[F_n=\{\omega\mid\omega_n=H\}.\]
Then, we take
\[\mathcal F=\sigma(F_n).\]

\section{Random Variables}

\begin{definition}
    A function $f:(X,\Sigma)\to(Y,T)$ is a \emph{measurable function} if for each $F\in T$ that
    \[f^{-1}(T)\in\Sigma.\]
    If $(Y,T)=\mathcal B$, then we will denote our function \emph{$\Sigma$-measurable}.

    Finally, denote the class of $\Sigma$-measurable functions as $m\Sigma$.
\end{definition}



\end{document}